\documentclass[10pt]{report}
\usepackage{a4wide}
\usepackage{epsfig}
\usepackage{array}
\usepackage{amsmath}
\usepackage{SIunits}
\usepackage{psfrag}
\usepackage{relsize}
\usepackage[section]{placeins}
\usepackage{listings}

\newif\ifpdf
\ifx\pdfoutput\undefined
  \pdffalse
\else
  \pdfoutput=1
  \pdftrue
\fi

\ifpdf
\pdfcompresslevel=9
\pdfinfo {
  /Title   (Qucs)
  /Subject (Technical Papers)
  /Author  (Stefan Jahn)
}
\fi

\makeatletter
\def\thickhrulefill{\leavevmode \leaders \hrule height 1pt\hfill \kern \z@}
\renewcommand{\maketitle}{\begin{titlepage}%
    \let\footnotesize\small
    \let\footnoterule\relax
    \parindent \z@
    \reset@font
    \null\vfil
    \vspace*{3cm}
    \begin{flushleft}
      \bf \huge \@title
    \end{flushleft}
    \par
    \hrule height 3pt
    \par
    \begin{flushright}
      \LARGE Technical Papers \par
    \end{flushright}
    \vskip 60\p@
    \vfill


    \begin{flushright}
      \Large \@author \par
    \end{flushright}

    \hrule height 3pt \par

\vspace*{24pt}

Copyright \copyright{} 2003 Michael Margraf 
\textless margraf@mwt.ee.tu-berlin.de\textgreater \par
Copyright \copyright{} 2003 Stefan Jahn 
\textless jahn@mwt.ee.tu-berlin.de\textgreater \par

\vspace*{12pt}

Permission is granted to copy, distribute and/or modify this document
under the terms of the GNU Free Documentation License, Version 1.1 or
any later version published by the Free Software Foundation.  A copy
of the license is included in the section entitled "GNU Free
Documentation License".

\vspace*{1cm}

  \end{titlepage}%
  \setcounter{footnote}{0}%
}
\makeatother

\author{Michael Margraf \\ Stefan Jahn}
\title{Qucs}
\date{}

\begin{document}

\maketitle

\tableofcontents

\setlength{\parindent}{0pt}
\newpage

\chapter*{Mathematical expressions and conversions}
\addcontentsline{toc}{chapter}{Mathematical expressions and conversions}

\section*{Scattering parameters}
\addcontentsline{toc}{section}{Scattering parameters}

\subsection*{Recalculating $\mathbf{50\ohm}$-S-parameters for arbitrary port impedances}
\addcontentsline{toc}{subsection}{Recalculating $50\ohm$-S-parameters for arbitrary port impedances}

During S-parameter simulation it is necessary to have all components
in a circuit normalized to the same impedance.  In the field of high
frequency techniques this is usually $50\ohm$.  In order to allow port
impedances other than $50\ohm$ in a simulation the following two step
process must be applied to the resulting S-parameter analysis.

\begin{equation}
\left[\underline{N}\right] = 
\left(\left[\underline{S}\right] - \left[\underline{R}\right]\right) \cdot
\left(\left[\underline{E}\right] - \left[\underline{R}\right] \cdot \left[\underline{S}\right]\right)^{-1}
\end{equation}

\begin{equation}
\underline{N}_{nm} = \underline{S}_{nm}\cdot \sqrt{\dfrac{Z^{m}}{Z^{n}}}\cdot
\dfrac{Z^{n} + Z_{0}}{Z^{m} + Z_{0}}
\end{equation}

With\\

\begin{tabular}{rll}
$Z_{0}$ & = & $50\ohm$\\& &\\
$\left[\underline{E}\right]$ & = &
$\begin{pmatrix}
1 & 0 & \ldots & 0\\
0 & 1 & \ldots & 0\\
\vdots & \vdots & \ddots & \vdots\\
0 & 0 & \ldots & 1\\
\end{pmatrix}$
identity matrix\\& &\\
$\left[\underline{S}\right]$ & = & original $50\ohm$-S-parameter matrix\\& &\\
$\left[\underline{N}\right]$ & = & recalculated scattering matrix\\& &\\
$\left[\underline{R}\right]$ & = &
$\begin{pmatrix}
\underline{r}(Z_{1}) & 0 & \ldots & 0\\
0 & \underline{r}(Z_{2}) & \ldots & 0\\
\vdots & \vdots & \ddots & \vdots\\
0 & 0 & \ldots & \underline{r}(Z_{n})\\
\end{pmatrix}$
reflection coefficient matrix\\& &\\
$\underline{r}(Z_{n})$ & = &
$\dfrac{Z_{n} - Z_{0}}{Z_{n} + Z_{0}}$
reflection coefficient of impedance at port n\\& &\\
\end{tabular}

And furthermore\\

\begin{tabular}{rll}
$\left[\underline{X}\right]^{-1}$ & = & 
inverted matrix of $\left[\underline{X}\right]$\\& &\\
$\underline{X}_{nm}$ & = & 
element of matrix $\left[\underline{X}\right]$ at row n and column m\\& &\\
\end{tabular}

\subsection*{Differential S-parameter ports}
\addcontentsline{toc}{subsection}{Differential S-parameter ports}

The implemented algorithm for the S-parameter analysis calculates
S-parameters in terms of the ground node.  In order to allow
differential S-parameters as well it is necessary to insert an ideal
impedance transformer with a turns ratio of 1:1 between the
differential port and the device under test.

\begin{figure}[ht]
\begin{center}
\includegraphics[width=12cm]{differential}
\end{center}
\caption{transformation of differential port into single ended port}
\label{fig:differential}
\end{figure}
\FloatBarrier

The S-parameter matrix of the inserted ideal transformer being a three
port device can be written as follows.

\begin{equation}
\begin{pmatrix}
S
\end{pmatrix}
= \dfrac{1}{3}\cdot
\begin{pmatrix}
1 & 2 & -2\\
2 & 1 & 2\\
-2 & 2 & 1\\
\end{pmatrix}
\end{equation}

This transformation can be applied to each S-parameter port in a
circuit regardless whether it is actually differential or not.

It is also possible to do the impedance transformation within this step
(for S-parameter ports with impedances different than $50\ohm$). This can
be done by using a transformer with an impedance ration of

\begin{equation}
r=T^2=\frac{50\ohm}{Z}
\end{equation}

With $Z$ being the S-parameter port impedance. The S-parameter matrix of
the inserted ideal transformer now writes as follows.

\begin{equation}
\begin{pmatrix}
S
\end{pmatrix}
= \dfrac{1}{2\cdot Z_0+Z}\cdot
\begin{pmatrix}
2\cdot Z_0-Z              & 2\cdot\sqrt{Z_0\cdot Z}  & -2\cdot\sqrt{Z_0\cdot Z}\\
2\cdot\sqrt{Z_0\cdot Z}   & Z                        & 2\cdot Z_0\\
-2\cdot\sqrt{Z_0\cdot Z}  & 2\cdot Z_0               & Z\\
\end{pmatrix}
\end{equation}

With $Z$ being the new S-parameter port impedance and $Z_0$ being $50\ohm$.

\section*{Scattering parameters of components}
\addcontentsline{toc}{section}{Scattering parameters of components}

\subsection*{Resistor}
\addcontentsline{toc}{subsection}{Resistor}

The scattering parameters of an ideal, ohmic resistor with resistance $R$ writes as follows.

\begin{equation}
S_{11} = S_{22} = \frac{R}{2\cdot Z_0+R} \\
\end{equation}
\begin{equation}
S_{12} = S_{21} = 1-S_{11} = \frac{2\cdot Z_0}{2\cdot Z_0+R}
\end{equation}

\subsection*{Capacitor}
\addcontentsline{toc}{subsection}{Capacitor}

The scattering parameters of an ideal capacitor with capacitance $C$ writes as follows.

\begin{equation}
S_{11} = S_{22} = \frac{1}{2\cdot Z_0\cdot j\omega C+1} \\
\end{equation}
\begin{equation}
S_{12} = S_{21} = 1-S_{11}
\end{equation}

\subsection*{Inductor}
\addcontentsline{toc}{subsection}{Inductor}

The scattering parameters of an ideal inductor with inductance $L$ writes as follows.

\begin{equation}
S_{11} = S_{22} = \frac{j\omega L}{2\cdot Z_0 + j\omega L} \\
\end{equation}
\begin{equation}
S_{12} = S_{21} = 1-S_{11}
\end{equation}

\subsection*{DC Block}
\addcontentsline{toc}{subsection}{DC Block}

A DC block is a capacitor with an infinite capacitance. The scattering parameters,
therefore, writes as follows.

\begin{equation}
\begin{pmatrix}
S
\end{pmatrix}
=
\begin{pmatrix}
0 & 1\\
1 & 0\\
\end{pmatrix}
\end{equation}

\subsection*{DC Feed}
\addcontentsline{toc}{subsection}{DC Feed}

A DC feed is an inductor with an infinite inductance. The scattering parameters,
therefore, writes as follows.

\begin{equation}
\begin{pmatrix}
S
\end{pmatrix}
=
\begin{pmatrix}
1 & 0\\
0 & 1\\
\end{pmatrix}
\end{equation}

\subsection*{Bias T}
\addcontentsline{toc}{subsection}{Bias T}

A bias T is a combination of a dc block and a dc feed (fig. \ref{fig:biast}).
The scattering parameters, therefore, writes as follows.

\begin{equation}
\begin{pmatrix}
S
\end{pmatrix}
=
\begin{pmatrix}
0 & 1 & 0\\
1 & 0 & 0\\
0 & 0 & 1\\
\end{pmatrix}
\end{equation}

\begin{figure}[ht]
\begin{center}
\includegraphics[width=3.5cm]{biast}
\end{center}
\caption{bias t}
\label{fig:biast}
\end{figure}
\FloatBarrier

\subsection*{Transformer}
\addcontentsline{toc}{subsection}{Transformer}

Using the port numbers depicted in fig. \ref{fig:trafo}, the scattering
parameters of an ideal transformer with voltage transformation ratio $T$
(number of turns) writes as follows.

\begin{equation}
S_{14} = S_{22} = S_{33} = S_{41} = \frac{1}{T^2+1}
\end{equation}
\begin{equation}
S_{12} = -S_{13} = S_{21} = -S_{24} = -S_{31} = S_{34} = -S_{42} = S_{43} = T\cdot S_{22}
\end{equation}
\begin{equation}
S_{11} = S_{23} = S_{32} = S_{44} = T\cdot S_{12}
\end{equation}

\begin{figure}[ht]
\begin{center}
\includegraphics[width=4cm]{trafo}
\end{center}
\caption{transformer}
\label{fig:trafo}
\end{figure}
\FloatBarrier

\subsection*{Symmetrical transformer}
\addcontentsline{toc}{subsection}{Symmetrical transformer}

Using the port numbers depicted in fig. \ref{fig:symtrafo}, the scattering
parameters of an ideal, symmetrical transformer with voltage transformation
ratio (number of turns) $T_1$ and $T_2$, respectively, writes as follows.

\begin{equation}
denom = T_1^2+T_2^2+T_1^2\cdot T_2^2
\end{equation}
\begin{eqnarray}
S_{11} = S_{66} = \frac{T_2^2}{denom}  &  \qquad S_{16} = S_{61} = 1-S_{11} \\
S_{44} = S_{55} = \frac{T_1^2}{denom}  &  \qquad S_{45} = S_{54} = 1-S_{44} \\
S_{22} = S_{33} = \frac{T_1^2\cdot T_2^2}{denom}  &  \qquad S_{23} = S_{32} = 1-S_{22} \\
\end{eqnarray}
\begin{equation}
S_{12} = S_{21} = -S_{13} = -S_{31} = -S_{26} = -S_{62} = S_{36} = S_{63}
       = \frac{T_1\cdot T_2^2}{denum}
\end{equation}
\begin{equation}
-S_{24} = -S_{42} = S_{25} = S_{52} = S_{34} = S_{43} = -S_{35} = -S_{53}
       = \frac{T_1^2\cdot T_2}{denum}
\end{equation}
\begin{equation}
-S_{14} = -S_{41} = S_{15} = S_{51} = S_{46} = S_{64} = -S_{56} = -S_{65}
       = \frac{T_1\cdot T_2}{denum}
\end{equation}

\begin{figure}[ht]
\begin{center}
\includegraphics[width=4cm]{symtrafo}
\end{center}
\caption{symmetrical transformer}
\label{fig:symtrafo}
\end{figure}
\FloatBarrier

\subsection*{Attenuator}
\addcontentsline{toc}{subsection}{Attenuator}

The scattering parameters of an ideal attenuator with attenuation $L$ (loss) in
reference to the impedance $Z_{ref}$ writes as follows.

\begin{equation}
r = \frac{Z_0-Z_{ref}}{Z_0+Z_{ref}}
\end{equation}
\begin{equation}
S_{11} = S_{22} = \frac{r\cdot(1-L^2)}{L^2-r^2}
\end{equation}
\begin{equation}
S_{12} = S_{21} = \frac{L\cdot(1-r^2)}{L^2-r^2}
\end{equation}

\subsection*{Isolator}
\addcontentsline{toc}{subsection}{Isolator}

An isolator is a one-way two-port, transporting incomming waves lossless from
the input (port 1) to the output (port 2), but consuming all waves flowing
into the output. With the reference impedance of the input $Z_1$ and the one
of the output $Z_2$, the scattering parameters of an ideal isolator writes as
follows.

\begin{equation}
S_{11} = \frac{Z_1-Z_0}{Z_1+Z_0}
\end{equation}
\begin{equation}
S_{12} = 0
\end{equation}
\begin{equation}
S_{22} = \frac{Z_2-Z_0}{Z_2+Z_0}
\end{equation}
\begin{equation}
S_{21} = \sqrt{1-(S_{11})^2}\cdot\sqrt{1-(S_{22})^2}
\end{equation}

\subsection*{Circulator}
\addcontentsline{toc}{subsection}{Circulator}

A circulator is a 3-port device, transporting incomming waves lossless from port 1
to port 2, from port 2 to port 3 and from port 3 to port 1. In all other
directions, there is no energy flow. With the reference impedances $Z_1$, $Z_2$ and
$Z_3$ for the ports 1, 2 and 3 the scattering matrix of an ideal circulator writes
as follows.

\begin{equation}
denum = 1-r_1\cdot r_2\cdot r_3
\end{equation}
\begin{equation}
r_1 = \frac{Z_0-Z_1}{Z_0+Z_1} \quad,\qquad 
r_2 = \frac{Z_0-Z_2}{Z_0+Z_2} \quad,\qquad 
r_3 = \frac{Z_0-Z_3}{Z_0+Z_3}
\end{equation}
\begin{equation}
S_{11} = \frac{r_2\cdot r_3 - r_1}{denum} \quad,\qquad 
S_{22} = \frac{r_1\cdot r_3 - r_2}{denum} \quad,\qquad 
S_{33} = \frac{r_1\cdot r_2 - r_3}{denum}
\end{equation}
\begin{equation}
S_{12} = \sqrt{\frac{Z_2}{Z_1}}\cdot\frac{Z_1+Z_0}{Z_2+Z_0}\cdot\frac{r_3\cdot(1-r_1^2)}{denum}
\quad,\qquad 
S_{13} = \sqrt{\frac{Z_3}{Z_1}}\cdot\frac{Z_1+Z_0}{Z_3+Z_0}\cdot\frac{1-r_1^2}{denum}
\end{equation}
\begin{equation}
S_{21} = \sqrt{\frac{Z_1}{Z_2}}\cdot\frac{Z_2+Z_0}{Z_1+Z_0}\cdot\frac{1-r_2^2}{denum}
\quad,\qquad 
S_{23} = \sqrt{\frac{Z_3}{Z_2}}\cdot\frac{Z_2+Z_0}{Z_3+Z_0}\cdot\frac{r_1\cdot(1-r_2^2)}{denum}
\end{equation}
\begin{equation}
S_{31} = \sqrt{\frac{Z_1}{Z_3}}\cdot\frac{Z_3+Z_0}{Z_1+Z_0}\cdot\frac{r_2\cdot(1-r_3^2)}{denum}
\quad,\qquad 
S_{32} = \sqrt{\frac{Z_2}{Z_3}}\cdot\frac{Z_3+Z_0}{Z_2+Z_0}\cdot\frac{1-r_3^2}{denum}
\end{equation}

\subsection*{Phase shifter}
\addcontentsline{toc}{subsection}{Phase shifter}

The scattering parameters of an ideal phase shifter with phase shift $\phi$ and
reference impedance $Z_{ref}$ writes as follows.

\begin{equation}
r = \frac{Z_0-Z_{ref}}{Z_0+Z_{ref}}
\end{equation}
\begin{equation}
S_{11} = S_{22} = \frac{r\cdot\left(\exp(j\cdot 2\phi)-1\right)}{1-r^2\cdot\exp(j\cdot 2\phi)}
\end{equation}
\begin{equation}
S_{12} = S_{21} = \frac{(1-r^2)\cdot\exp(j\cdot\phi)}{1-r^2\cdot\exp(j\cdot 2\phi)}
\end{equation}


\subsection*{Gyrator}
\addcontentsline{toc}{subsection}{Gyrator}

A gyrator is an impedance inverter. Thus, for example, it converts a capacitance into
an inductance and vice versa. The scattering matrix of an ideal gyrator with the
ratio $R$ writes as follows.

\begin{equation}
r = \frac{R}{Z_0} = \frac{1}{G\cdot Z_0}
\end{equation}
\begin{equation}
S_{11} = S_{22} = S_{33} = S_{44} = \frac{R^2}{4\cdot Z_0^2 + R^2} = \frac{r^2}{r^2+4}
\end{equation}
\begin{equation}
S_{14} = S_{23} = S_{32} = S_{41} = 1-S_{11}
\end{equation}
\begin{equation}
S_{12} = -S_{13} = -S_{21} = S_{24} = S_{31} = -S_{34} = -S_{42} = S_{43} = \frac{2\cdot r}{r^2+4}
\end{equation}

\subsection*{Voltage and current sources}
\addcontentsline{toc}{subsection}{Voltage and current sources}

All voltage sources (AC and DC) are short circuits and therefore their S-parameter
matrix equals the one of the DC block. All current sources are open circuits and
therefore their S-parameter matrix equals the one of the DC feed.

\subsection*{Controlled voltage and current sources}
\addcontentsline{toc}{subsection}{Controlled voltage and current sources}

The scattering matrix of the voltage controlled current source depicted in fig.
\ref{fig:xCCS} (left) writes as follows ($\tau$ is time delay).

\begin{equation}
S_{11} = S_{22} = S_{33} = S_{44} = 1
\end{equation}
\begin{equation}
S_{12} = S_{13} = S_{14} = S_{23} = S_{32} = S_{41} = S_{42} = S_{43} = 0
\end{equation}
\begin{equation}
S_{21} = S_{34} = 2\cdot G\cdot \exp(j\pi-j\omega\tau)
\end{equation}
\begin{equation}
S_{24} = S_{31} = 2\cdot G\cdot \exp(-j\omega\tau)
\end{equation}

\begin{figure}[ht]
\begin{center}
\includegraphics[width=8cm]{xCCS}
\end{center}
\caption{voltage controlled current source (left) and current controlled current source (right)}
\label{fig:xCCS}
\end{figure}
\FloatBarrier

The scattering matrix of the current controlled current source depicted in fig.
\ref{fig:xCCS} (right) writes as follows ($\tau$ is time delay).

\begin{equation}
S_{14} = S_{22} = S_{33} = S_{41} = 1
\end{equation}
\begin{equation}
S_{11} = S_{12} = S_{13} = S_{23} = S_{32} = S_{42} = S_{43} = S_{44} = 0
\end{equation}
\begin{equation}
S_{21} = S_{34} = G\cdot \exp(j\pi-j\omega\tau)
\end{equation}
\begin{equation}
S_{24} = S_{31} = G\cdot \exp(-j\omega\tau)
\end{equation}

The scattering matrix of the voltage controlled voltage source depicted in fig.
\ref{fig:xCVS} (left) writes as follows ($\tau$ is time delay).

\begin{equation}
S_{11} = S_{23} = S_{32} = S_{44} = 1
\end{equation}
\begin{equation}
S_{12} = S_{13} = S_{14} = S_{22} = S_{33} = S_{41} = S_{42} = S_{43} = 0
\end{equation}
\begin{equation}
S_{21} = S_{34} = G\cdot \exp(-j\omega\tau)
\end{equation}
\begin{equation}
S_{24} = S_{31} = G\cdot \exp(j\pi-j\omega\tau)
\end{equation}

\begin{figure}[ht]
\begin{center}
\includegraphics[width=8cm]{xCVS}
\end{center}
\caption{voltage controlled voltage source (left) and current controlled voltage source (right)}
\label{fig:xCVS}
\end{figure}
\FloatBarrier

The scattering matrix of the current controlled voltage source depicted in fig.
\ref{fig:xCVS} (right) writes as follows ($\tau$ is time delay).

\begin{equation}
S_{14} = S_{23} = S_{32} = S_{41} = 1
\end{equation}
\begin{equation}
S_{11} = S_{12} = S_{13} = S_{22} = S_{33} = S_{42} = S_{43} = S_{44} = 0
\end{equation}
\begin{equation}
S_{21} = S_{34} = \frac{G}{2}\cdot \exp(-j\omega\tau)
\end{equation}
\begin{equation}
S_{24} = S_{31} = \frac{G}{2}\cdot \exp(j\pi-j\omega\tau)
\end{equation}

\subsection*{Transmission Line}
\addcontentsline{toc}{subsection}{Transmission Line}

The scattering matrix of an ideal, lossless transmission line with impedance $Z$
and electrical length $L$ writes as follows.

\begin{equation}
r = \frac{Z-Z_0}{Z+Z_0}
\end{equation}
\begin{equation}
p = \exp(-j\omega\frac{L}{c})
\end{equation}
\begin{equation}
S_{11} = S_{22} = \frac{r\cdot(1-p^2)}{1-r^2\cdot p^2} \quad,\qquad
S_{12} = S_{21} = \frac{p\cdot(1-r^2)}{1-r^2\cdot p^2}
\end{equation}

With $c$=299 792 458 m/s being the vacuum light velocity.

\end{document}
