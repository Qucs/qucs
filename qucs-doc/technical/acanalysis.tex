%
% This document contains the chapter about AC analysis.
%
% Copyright (C) 2003, 2004, 2005, 2006 Stefan Jahn <stefan@lkcc.org>
% Copyright (C) 2004, 2006 Michael Margraf <Michael.Margraf@alumni.TU-Berlin.DE>
%
% Permission is granted to copy, distribute and/or modify this document
% under the terms of the GNU Free Documentation License, Version 1.1
% or any later version published by the Free Software Foundation.
%

\chapter{AC Analysis}
%\addcontentsline{toc}{chapter}{AC Analysis}
\label{sec:acMNA}

The AC analysis is a small signal analysis in the frequency domain.
Basically this type of simulation uses the same algorithms as the DC
analysis (section \ref{sec:MNA} on page \pageref{sec:MNA}).  The AC
analysis is a linear modified nodal analysis.  Thus no iterative
process is necessary.  With the Y-matrix of the components, i.e. now a
complex matrix, and the appropriate extensions it is necessary to
solve the equation system \eqref{eq:acMNA} similar to the (linear) DC
analysis.
\begin{equation}
\label{eq:acMNA}
\left[A\right] \cdot \left[x\right] = \left[z\right]
\;\;\;\; \textrm{ with } \;\;\;\;
A =
\begin{bmatrix}
Y & B\\
C & D
\end{bmatrix}
\end{equation}

Non-linear components have to be linearized at the DC bias point.
That is, before an AC simulation with non-linear components can
be performed, a DC simulation must be completed successfully.
Then, the MNA stamp of the non-linear components equals their
entries of the Jacobian matrix, which was already computed
during the DC simulation. In addition to this real-valued elements,
a further stamp has to be applied: The Jacobian matrix of the
non-linear charges multiplied by $j\omega$ (see also section
\ref{sec:eqn_def_models}).
