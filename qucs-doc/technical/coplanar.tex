%
% This document contains the chapter about coplanar components.
%
% Copyright (C) 2004, 2005 Vincent Habchi, F5RCS <10.50@free.fr>
% Copyright (C) 2004, 2005 Stefan Jahn <stefan@lkcc.org>
%
% Permission is granted to copy, distribute and/or modify this document
% under the terms of the GNU Free Documentation License, Version 1.1
% or any later version published by the Free Software Foundation.
%

\chapter{Coplanar components}
%\addcontentsline{toc}{chapter}{Coplanar components}

\section{Coplanar waveguides (CPW)}
%\addcontentsline{toc}{section}{Coplanar waveguides (CPW)}

\subsection{Definition}
%\addcontentsline{toc}{subsection}{Definition}

A \emph{coplanar line} is a structure in which all the conductors
supporting wave propagation are located on the same plane,
i.e. generally the top of a dielectric substrate.  There exist two
main types of coplanar lines: the first, called \emph{coplanar
waveguide (CPW)}, that we will study here, is composed of a median
metallic strip separated by two narrow slits from a infinite ground
plane, as may be seen on the figure below.

\begin{figure}[ht]
\begin{center}
\includegraphics[width=0.9\linewidth]{cpline}
\end{center}
\caption{coplanar waveguide line}
\label{fig:CoplanarLine}
\end{figure}
\FloatBarrier

The characteristic dimensions of a CPW are the central strip width $W$
and the width of the slots $s$.  The structure is obviously
symmetrical along a vertical plane running in the middle of the
central strip.

\begin{figure}[ht]
\begin{center}
\includegraphics[width=0.5\linewidth]{cpscheme}
\end{center}
\label{fig:CoplanarScheme}
\end{figure}
\FloatBarrier

The other coplanar line, called a \emph{coplanar slot (CPS)} is the
complementary of that topology, consisting of two strips running side
by side.

\subsection{Quasi-static analysis by conformal mappings}
%\addcontentsline{toc}{subsection}{Quasi-static analysis by conformal mappings}

A CPW can be quasi-statically analysed by the use of
\emph{conformal mappings}.  Briefly speaking, it consists in
transforming the geometry of the PCB into another conformation, whose
properties make the computations straightforward.  The interested
reader can consult the pp. 886~-~910 of \cite{Collin} which has a
correct coverage of both the theoretical and applied methods.  The
French reader interested in the mathematical arcanes involved is
referred to the second chapter of \cite{Mir} (which may be out of
print nowadays), for an extensive review of all the theoretical
framework.  The following analysis is mainly borrowed from
\cite{Gupta1}, pp.~375 \emph{et seq}. with additions from
\cite{Collin}.

\addvspace{12pt}

The CPW of negligible thickness located on top of an infinitely deep
substrate, as shown on the left of the figure below, can be mapped
into a parallel plate capacitor filled with dielectric $ABCD$ using
the conformal function:
\begin{equation}
w = \int_{z_0}^{z}\dfrac{dz}{\sqrt{(z-W/2)(z-W/2-s)}}.
\end{equation}

\begin{figure}[ht]
\begin{center}
\includegraphics[width=0.7\linewidth]{cpconform}
\end{center}
\label{fig:CoplanarConformal}
\end{figure}
\FloatBarrier

To further simplify the analysis, the original dielectric boundary is
assumed to constitute a magnetic wall, so that $BC$ and $AD$ become
magnetic walls too and there is no resulting fringing field in the
resulting capacitor.  With that assumption, the capacitance per unit
length is merely the sum of the top (air filled) and bottom
(dielectric filled) partial capacitances.  The latter is given by:
\begin{equation}
C_d = 2\cdot \varepsilon_0\cdot \varepsilon_r\cdot \dfrac{K(k_1)}{K'(k_1)}
\end{equation}

while the former is:
\begin{equation}
C_a = 2\cdot \varepsilon_0\cdot \dfrac{K(k_1)}{K'(k_1)}
\end{equation}

In both formulae $K(k)$ and $K'(k)$ represent the complete elliptic
integral of the first kind and its complement, and $k_1=
\tfrac{W}{W+2s}$.

\addvspace{12pt}

$K(k)$ (and $K'(k)$) can be computed easily by using the \emph{arithmetic-geometric mean} method \cite{Abramowitz}: starting with
\begin{equation}
a_0 = 1, \quad b_0 = \sqrt{1 -k^2}
\end{equation}
the following recursion formulae are iterated
\begin{equation}
a_n = \dfrac{1}{2}(a_{n-1}+b_{n-1}), \quad b_n = \sqrt{a_{n-1}b_{n-1}}, \quad c_n=a_{n-1}-b_{n-1}
\end{equation}
until $c_N=0$ within a predetermined accuracy.\\
$K(k)$ is then found from
\begin{equation}
K(k) = \dfrac{\pi}{2 a_N}
\end{equation}

\addvspace{12pt}

When only the the $K/K'$ ratio is needed it can be approximated
efficiently through the following formulae:
\begin{equation}
\dfrac{K(k)}{K'(k)} =
\dfrac{\pi}{\ln\left(2\tfrac{1+\sqrt{k'}}{1-\sqrt{k'}}\right)}
\;\;\;\; \textrm{ for } \;\;\;\;
0 \le k \le \dfrac{1}{\sqrt{2}}
\end{equation}
\begin{equation}
\dfrac{K(k)}{K'(k)} =
\dfrac{\ln\left(2\tfrac{1+\sqrt{k}}{1-\sqrt{k}}\right)}{\pi}
\;\;\;\; \textrm{ for } \;\;\;\;
\dfrac{1}{\sqrt{2}} \le k \le 1
\end{equation}

with $k'$ being the complementary modulus: $k'=\sqrt{1-k^2}$.
This formula was first derived in \cite{Hilberg}. According to
\cite{Collin} the accuracy of the above formulae is close
to $10^{-5}$, \cite{Gupta1} claims it to be $3\cdot10^{-6}$ while comparison
with GNU Octave using the $ellipke()$ function shows a maximum relative error
of about $2\cdot10^{-6}$ . More accurate approximations can be found in 
\cite{Hilberg} and \cite{Abbott} but the above formulae can already be 
considered as exact for any practical purpose.

\addvspace{12pt}

The total line capacitance is thus the sum of $C_d$ and $C_a$.  The
effective permittivity is therefore:
\begin{equation}
\varepsilon_{re} = \dfrac{\varepsilon_r+1}{2}
\end{equation}

and the impedance:
\begin{equation}
Z=\dfrac{30\pi}{\sqrt{\varepsilon_{re}}}\cdot\dfrac{K'(k_1)}{K(k_1)}
\label{eq:CoplanarZL}
\end{equation}

\begin{figure}[ht]
\begin{center}
\psfrag{impedance ZL in Ohm}{$\mathrm{\text{impedance }Z_{L}\text{ in }[\ohm]}$}
\psfrag{normalised strip width W/s}{normalised strip width W/s}
\includegraphics[width=0.95\linewidth]{coplanarzl}
\end{center}
\caption{characteristic impedance as approximated by eq. \eqref{eq:CoplanarZL} for $\varepsilon_{r}$ = $1.0$ (air), $3.78$ (quartz) and $9.5$ (alumina)}
\label{fig:coplanarzl}
\end{figure}
\FloatBarrier

In practical cases, the substrate has a finite thickness $h$.  To
carry out the analysis of this conformation, a preliminary conformal
mapping transforms the finite thickness dielectric into an infinite
thickness one.  Only the effective permittivity is altered; it
becomes:
\begin{equation}
\varepsilon_{re}=1+\dfrac{\varepsilon_r-1}{2}\cdot\dfrac{K(k_2)}{K'(k_2)}\cdot\dfrac{K'(k_1)}{K(k_1)}
\end{equation}

where $k_1$ is given above and
\begin{equation}
k_2=\dfrac{\sinh\left(\dfrac{\pi W}{4h}\right)}{\sinh\left(\dfrac{\pi\cdot\left(W+2s\right)}{4h}\right)}.
\end{equation}

Finally, let us consider a CPW over a finite thickness dielectric
backed by an infinite ground plane.  In this case, the quasi-TEM wave
is an hybrid between microstrip and true CPW mode.  The equations then
become:
\begin{equation}
\varepsilon_{re} =1 + q\cdot\left(\varepsilon_r - 1\right)
\end{equation}

where $q$, called \emph{filling factor} is given by:
\begin{equation}
q = \dfrac{\dfrac{K(k_3)}{K'(k_3)}}{\dfrac{K(k_1)}{K'(k_1)}+\dfrac{K(k_3)}{K'(k_3)}}
\end{equation}

and
\begin{equation}
k_3 = \dfrac{\tanh\left(\dfrac{\pi W}{4h}\right)}{\tanh\left(\dfrac{\pi\cdot\left(W+2s\right)}{4h}\right)}
\end{equation}

The impedance of this line amounts to:
\begin{equation}
Z=\dfrac{60\pi}{\sqrt{\varepsilon_{re}}}\cdot\dfrac{1}{\dfrac{K(k_1)}{K'(k_1)}+\dfrac{K(k_3)}{K'(k_3)}}
\end{equation}

\subsection{Effects of metalization thickness}
%\addcontentsline{toc}{subsection}{Effects of metalization thickness}

In most practical cases, the strips are very thin, yet their thickness
cannot be entirely neglected.  A first order correction to take into
account the non-zero thickness of the conductor is given by
\cite{Gupta1}:
\begin{equation}
s_e = s - \Delta
\end{equation}

and
\begin{equation}
W_e = W + \Delta
\end{equation}

where
\begin{equation}
\Delta = \dfrac{1.25t}{\pi}\cdot\left(1 + \ln\left(\dfrac{4 \pi W}{t}\right)\right)
\end{equation}

In the computation of the impedance, both the $k_1$ and the effective
dielectric constant are affected, wherefore $k_1$ must be substituted
by an ``effective'' modulus $k_e$, with:
\begin{equation}
k_e = \dfrac {W_e}{W_e + 2 s_e} \approx k_1 + \left(1 - k_1^2\right)\cdot \dfrac{\Delta}{2s}
\end{equation}

and
\begin{equation}
\varepsilon_{re}^{t} = \varepsilon_{re} - \dfrac{0.7\cdot\left(\varepsilon_{re} - 1\right)\cdot\dfrac{t}{s}}{\dfrac{K(k_1)}{K'(k_1)} + 0.7\cdot \dfrac{t}{s}}
\end{equation}

\subsection{Effects of dispersion}
%\addcontentsline{toc}{subsection}{Effects of dispersion}

The effects of dispersion in CPW are similar to those encountered in
the microstrip lines, though the net effect on impedance is somewhat
different.  \cite{Gupta1} gives a closed form expression to compute
$\varepsilon_{re}(f)$ from its quasi-static value:
\begin{equation}
\sqrt{\varepsilon_{re}(f)}=\sqrt{\varepsilon_{re}(0)} + \dfrac{\sqrt{\varepsilon_r} - \sqrt{\varepsilon_{re}(0)}}{1+G\cdot\left(\dfrac{f}{f_{TE}}\right)^{-1.8}}
\end{equation}

where:
\begin{align}
G &= e^{u\cdot \ln\left(\tfrac{W}{s}\right)+v}\\
u &= 0.54 - 0.64p + 0.015p^2\\
v &= 0.43-0.86p+0.54p^2\\
p &= \ln\left(\tfrac{W}{h}\right)
\end{align}

and $f_{TE}$ is the cut-off frequency of the ${TE}_{0}$ mode, defined by:
\begin{equation}
f_{TE} = \dfrac{c}{4h\cdot\sqrt{\varepsilon_r - 1}}.
\end{equation}

This dispersion expression was first reported by \cite{Frankel} and
has been reused and extended in \cite{Gevorgian}.  The accuracy of
this expression is claimed to be better than 5\% for $0.1 \le W/h \le
5$, $0.1 \le W/s \le 5$, $1.5 \le \varepsilon_r \le 50$ and $0 \le
f/f_{TE} \le 10$.

\subsection{Evaluation of losses}
%\addcontentsline{toc}{subsection}{Evaluation of losses}

As for microstrip lines, the losses in CPW results of at least two
factors: a dielectric loss $\alpha_d$ and conductor losses
$\alpha_c^{CW}$.  The dielectric loss $\alpha_d$ is identical to the
microstrip case, see eq. \eqref{eq:alpha_d} on page
\pageref{eq:alpha_d}.

\addvspace{12pt}

The $\alpha_c^{CW}$ part of the losses is more complex to evaluate.  As
a general rule, it might be written:
\begin{equation}
\alpha_c^{CW} = 0.023\cdot \dfrac{R_s}{Z_{0cp}}\left[\dfrac{\partial Z^a_{0cp}}{\partial s} - \dfrac{\partial Z^a_{0cp}}{\partial W} - \dfrac{\partial Z^a_{0cp}}{\partial t}\right]
\textrm{ in dB/unit length }
\end{equation}

where $Z^a_{0cp}$ stands for the impedance of the coplanar waveguide
with air as dielectric and $R_s$ is the surface resistivity of the
conductors (see eq. \eqref{eq:HandJRs} on page \pageref{eq:HandJRs}).

\addvspace{12pt}

Through a direct approach evaluating the losses by conformal mapping
of the current density, one obtains \cite{Gupta1}, first reported in
\cite{Owyang} and finally applied to coplanar lines by \cite{Ghione1}:
\begin{equation}
\begin{split}
\alpha_c^{CW} = &\; \dfrac{R_s\cdot \sqrt{\varepsilon_{re}}}{480 \pi\cdot K(k_1)\cdot K'(k_1)\cdot \left(1 - k_1^2\right)} \cdot\\
& \left(\dfrac{1}{a}\left[\pi + \ln \dfrac {8 \pi a\cdot \left(1-k_1\right)}{t\cdot\left(1+k_1\right)}\right]+\dfrac{1}{b}\left[\pi + \ln \dfrac {8 \pi b\cdot \left(1-k_1\right)}{t\cdot\left(1+k_1\right)}\right]\right)
\end{split}
\end{equation}

In the formula above, $a = W/2, b = s + W/2$ and it is assumed that
$t~>~3 \delta, t \ll W$ and $t \ll s$.

\subsection{S- and Y-parameters of the single coplanar line}
%\addcontentsline{toc}{subsection}{S- and Y-parameters of the single coplanar line}

The computation of the coplanar waveguide lines S- and Y-parameters is
equal to all transmission lines (see section
section \ref{sec:tl_yparameter} on page \pageref{sec:tl_yparameter}).

\section{Coplanar waveguide open}
%\addcontentsline{toc}{section}{Coplanar waveguide open}

The behaviour of an open circuit as shown in fig. \ref{fig:CPWopen} is
very similar to that in a microstrip line; that is, the open circuit
is capacitive in nature.

\begin{figure}[ht]
\begin{center}
\includegraphics[width=0.6\linewidth]{cpopen}
\end{center}
\caption{coplanar waveguide open-circuit}
\label{fig:CPWopen}
\end{figure}
\FloatBarrier

A very simple approximation for the equivalent length extension
$\Delta l$ associated with the fringing fields has been given by
K.Beilenhoff \cite{Beilenhoff}.
\begin{equation}
\Delta l_{open} = \dfrac{C_{open}}{C'} \approx \dfrac{W + 2s}{4}
\end{equation}

For the open end, the value of $\Delta l$ is not influenced
significantly by the metalization thickness and the gap width $g$
when $g > W + 2s$.  Also, the effect of frequency and aspect ration $W
/ (W + 2s)$ is relatively weak.  The above approximation is valid for
$0.2 \le W / (W + 2s) \le 0.8$.

\addvspace{12pt}

The open end capacitance $C_{open}$ can be written in terms of the
capacitance per unit length and the wave resistance.
\begin{equation}
C_{open} = C'\cdot \Delta l_{open} = \dfrac{\sqrt{\varepsilon_{r,eff}}}{c_0\cdot Z_L} \cdot \Delta l_{open}
\end{equation}

\section{Coplanar waveguide short}
%\addcontentsline{toc}{section}{Coplanar waveguide short}

There is a similar simple approximation for a coplanar waveguide
short-circuit, also given in \cite{Beilenhoff}.  The short circuit is
inductive in nature.

\begin{figure}[ht]
\begin{center}
\includegraphics[width=0.6\linewidth]{cpshort}
\end{center}
\caption{coplanar waveguide short-circuit}
\label{fig:CPWshort}
\end{figure}
\FloatBarrier

The equivalent length extension $\Delta l$ associated with the
fringing fields is
\begin{equation}
\label{eq:CPWshort}
\Delta l_{short} = \dfrac{L_{short}}{L'} \approx \dfrac{W + 2s}{8}
\end{equation}

Equation \eqref{eq:CPWshort} is valid when the metalization thickness
$t$ does not become too large ($t < s/3$).

\addvspace{12pt}

The short end inductance $L_{short}$ can be written in terms of the
inductance per unit length and the wave resistance.
\begin{equation}
L_{short} = L'\cdot \Delta l_{short} = \dfrac{\sqrt{\varepsilon_{r,eff}}\cdot Z_L}{c_0} \cdot \Delta l_{short}
\end{equation}

\addvspace{12pt}

According to W.J.Getsinger \cite{Getsinger6} the CPW short-circuit
inductance per unit length can also be modeled by
\begin{equation}
L_{short} = \dfrac{2}{\pi}\cdot \varepsilon_0 \cdot \varepsilon_{r,eff} \cdot \left(W + s\right)\cdot Z_L^2\cdot \left(1 - \text{sech} \left(\dfrac{\pi\cdot Z_{F0}}{2\cdot Z_L\cdot \sqrt{\varepsilon_{r,eff}}}\right)\right)
\end{equation}

based on his duality \cite{Getsinger5} theory.

\section{Coplanar waveguide gap}
%\addcontentsline{toc}{section}{Coplanar waveguide gap}

According to W.J.Getsinger \cite{Getsinger5} a coplanar series gap
(see fig. \ref{fig:CPWgap}) is supposed to be the dual problem of the
inductance of a connecting strip between twin strip lines.

\begin{figure}[ht]
\begin{center}
\includegraphics[width=0.45\linewidth]{cpgap}
\end{center}
\caption{coplanar waveguide series gap}
\label{fig:CPWgap}
\end{figure}
\FloatBarrier

The inductance of such a thin strip with a width $g$ and the length
$W$ is given to a good approximation by
\begin{equation}
L = \dfrac{\mu_0\cdot W}{2\pi}\cdot\left(p - \sqrt{1 + p^2} + \ln{\left(\dfrac{1 + \sqrt{1 + p^2}}{p}\right)}\right)
\end{equation}

where $p = g/4W$ and $g,W \ll \lambda$.  Substituting this inductance
by its equivalent capacitance of the gap in CPW yields
\begin{equation}
\begin{split}
C &= L \cdot \dfrac{4\cdot \varepsilon_{r,eff}}{Z_{F0}^2}\\
  &= \dfrac{2\cdot\varepsilon_0\cdot\varepsilon_{r,eff}\cdot W}{\pi}\cdot\left(p - \sqrt{1 + p^2} + \ln{\left(\dfrac{1 + \sqrt{1 + p^2}}{p}\right)}\right)
\end{split}
\end{equation}

\section{Coplanar waveguide step}
%\addcontentsline{toc}{section}{Coplanar waveguide step}

The coplanar step discontinuity shown in figure \ref{fig:CPWstep} has
been analysed by C. Sinclair \cite{Sinclair}.

\begin{figure}[ht]
\begin{center}
\includegraphics[width=0.8\linewidth]{cpstep}
\end{center}
\caption{coplanar waveguide impedance step and equivalent circuit}
\label{fig:CPWstep}
\end{figure}
\FloatBarrier

The symmetric step change in width of the centre conductor is
considered to have a similar equivalent circuit as a step of a
parallel plate guide - this is a reasonable approximation to the CPW
step as in the CPW the majority of the field is between the inner and
outer conductors with some fringing.

\addvspace{12pt}

The actual CPW capacitance can be expressed as
\begin{equation}
C = \overline{x}\cdot \dfrac{\varepsilon_0}{\pi}\cdot\left(\dfrac{\alpha^2 + 1}{\alpha}\cdot \ln{\left(\dfrac{1 + \alpha}{1 - \alpha}\right)} - 2\cdot \ln{\left(\dfrac{4\cdot \alpha}{1 - \alpha^2}\right)}\right)
\end{equation}

where
\begin{equation}
\alpha = \dfrac{s_1}{s_2}, \alpha < 1
\;\;\;\; \textrm{ and } \;\;\;\;
\overline{x} = \dfrac{x_1 + x_2}{2}
\end{equation}

The capacitance per unit length equivalence yields
\begin{equation}
x_1 = \dfrac{C'\left(W_1, s_1\right)\cdot s_1}{\varepsilon_0}
\;\;\;\; \textrm{ and } \;\;\;\;
x_2 = \dfrac{C'\left(W_2, s_2\right)\cdot s_2}{\varepsilon_0}
\end{equation}

with
\begin{equation}
C' = \dfrac{\sqrt{\varepsilon_{r,eff}}}{c_0\cdot Z_L}
\end{equation}

The average equivalent width $\overline{x}$ of the parallel plate
guide can be adjusted with an expression that uses weighted average of
the gaps $s_1$ and $s_2$.  The final expression has not been discussed
in \cite{Sinclair}.  The given equations are validated over the
following ranges: $2 < \varepsilon_r < 14$, $h > W + 2s$ and $f <
40\giga\hertz$.

\addvspace{12pt}

The Z-parameters of the equivalent circuit depicted in
fig. \ref{fig:CPWstep} are
\begin{equation}
Z_{11} = Z_{21} = Z_{12} = Z_{22} = \dfrac{1}{j\omega C}
\end{equation}

The MNA matrix representation for the AC analysis can be derived from
the Z-parameters in the following way.
\begin{equation}
\begin{bmatrix}
 . & .  & 1  & 0 \\
 . & .  &  0 & 1 \\
-1 &  0 & Z_{11} & Z_{12}\\
 0 & -1 & Z_{21} & Z_{22}
\end{bmatrix}
\cdot
\begin{bmatrix}
V_{1}\\
V_{2}\\
I_{in}\\
I_{out}
\end{bmatrix}
=
\begin{bmatrix}
I_{1}\\
I_{2}\\
0\\
0
\end{bmatrix}
\end{equation}

The above expanded representation using the Z-parameters is necessary
because the Y-parameters are infinite.  During DC analysis the
equivalent circuit is a voltage source between both terminals with
zero voltage.

\addvspace{12pt}

The S-parameters of the topology are
\begin{align}
S_{11} = S_{22} &= -\dfrac{Z_0}{2 Z + Z_0}\\
S_{12} = S_{21} &= 1 + S_{11} = \dfrac{2 Z}{2 Z + Z_0}
\end{align}
