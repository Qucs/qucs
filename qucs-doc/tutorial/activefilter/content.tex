%
% Tutorial -- Modelling Operational Amplifiers
%
% Copyright (C) 2006, 2007 Mike Brinson <mbrin72043@yahoo.co.uk>
%
% Permission is granted to copy, distribute and/or modify this document
% under the terms of the GNU Free Documentation License, Version 1.1
% or any later version published by the Free Software Foundation.
%

% redefine subfigure caption
\renewcommand{\thesubfigure}{\thefigure(\alph{subfigure})}
\makeatletter
  \renewcommand{\@thesubfigure}{\thesubfigure:\space}
  \renewcommand{\p@subfigure}{}
\makeatother

% redefine subtable caption
\renewcommand{\thesubtable}{\thetable(\alph{subtable})}
\makeatletter
  \renewcommand{\@thesubtable}{\thesubtable:\space}
  \renewcommand{\p@subtable}{}
\makeatother

\tutsection{Introduction}


\tutsection{End note}
While writing this tutorial I have tried to demonstrate how practical models of
operational amplifiers can be constructed using basic electronic concepts and
the range of Qucs built-in components.  The modular OP AMP macromodel was
deliberately chosen as the foundation for the tutorial for two reasons; firstly
Qucs is mature enough to easily simulate such models, and secondly the
parameters which determine the operation of the macromodel can be be calculated
directly from information provided on device data sheets. Recent modelling
development by the Qucs team has concentrated on improving the SPICE to Qucs
conversion facilities. This work has had a direct impact on Qucs ability to
import and simulate manufacturers OP AMP models. The tutorial upgrade explains
how SPICE Boyle type OP AMP macromodels can be converted to work with Qucs. The
Qucs OP AMP library (OpAmps) has been extended to include models for a range of
popular 8 pin DIL devices. If you require a model with a specific specification
that is not 
modelled by an available macromodel then adding extra functionality may be the
only way forward.  Two procedures for extending models are outlined in the
tutorial upgrade. Much work still remains to be done before Qucs can simulate a
wide range of the macromodels published by device manufacturers.  With the
recent addition of subcircuit/component equations to Qucs it is now possible to
write generalised macromodel macros for OP AMPs.  However, before this can be
done time is required to fully test the features that Stefan and Michael have
recently added to Qucs release 0.0.11. This topic and the modelling of other OP
AMP properties such as noise will be the subject of a further OP AMP tutorial
update sometime in the future.  My thanks to David Faulkner for all his help and
support during the period we were working on a number of the concepts that form
part of the basis of this tutorial. Once again a special thanks to Michael
Margraf and Stefan Jahn for all their help and encouragement over the period
that I 
have been writing this tutorial and testing the many examples it includes.