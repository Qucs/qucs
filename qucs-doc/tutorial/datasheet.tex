%
% Tutorial -- Understanding RF Data Sheet Parameters
%
% Copyright (C) 2005 Thierry Scordilis <thierry.scordilis@free.fr>
%
% Permission is granted to copy, distribute and/or modify this document
% under the terms of the GNU Free Documentation License, Version 1.1
% or any later version published by the Free Software Foundation.
%

\textit{\ldots prepared by Norman E.Dye from Motorola RF Division : AN 1107\footnote{This note could be found on old application notes databook from Motorola, if you have one keep them, it is a real treasure.}. Since this AN is essential to our topics, it is good to make a small reference to it. All AN from Motorola are a reference is this field. This chapter is only an extract, but the main points are hilighted herein.  \ldots \\ The author.}

\section{Introduction}
Data sheets are often the sole source of information about the
capability and characteristics of a product. This is particularly true
of unique RF semiconductor devices that are used by equipment
designers all over the world. Because the circuit designer often
cannot talk directly with the factory, he relies on the data sheet for
his device information. And for RF devices, many of the specifications
are unique in themselves. Thus it is important that the user and the
manufacturer of RF products speak a common language, what the
semiconductor manufacturer says about his RF device is understood
fully by the circuit designer.

\bigskip 

This paper reviews RF transistor and amplifier module parameters from
maximum ratings to functional characteristics.  It is divided into
five basic sections:

\begin{enumerate}
\item DC specifications,
\item power transistors,
\item low power transistor,
\item power modules,
\item linear modules.
\end{enumerate} 

Comments are made about critical specifications about how values are
determined and what are their significance.

\section{DC specifications}

Basically, RF transistors are characterized by two types of
parameters: DC and functional. The "DC" specs consist of breakdown
voltage, leakage current, $h_{FE}$ ( DC $\beta$ ) and capacitances,
while the functional specs cover gain, ruggedness, noise figure,
$Z_{in}$ and $Z_{out}$, S parameters, distortion, etc \ldots. Thermal
characteristics do not fall cleanly into either category since thermal
resistance and power dissipation can be either DC or AC. Thus we will
treat the spec of thermal resistance as a special specification and
give it its own heading called "thermal characteristics".

\section{Maximum ratings and thermal characteristics}
